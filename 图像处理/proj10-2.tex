\documentclass{../source/Experiment}

\major{信息工程}
\name{}
\title{形态学和其他集合运算}
\stuid{}
\college{信息与电子工程学院}
\date{\today}
\lab{}
\course{数字图像处理}
\instructor{李东晓}
\grades{}
\expname{形态学和其他集合运算}
\exptype{设计验证}
\partner{}
\begin{document}
\makecover
\section{实验任务}

本次选择的是PROJECT-10-02题目。

\begin{enumerate}
    \item 编写一个程序实现基本全局阈值处理,输出应该为一个二值图像。
    \item 下载图10.38(a),进行全局阈值处理,其结果应该与书上的一样。
\end{enumerate}

\section{算法设计}
算法实现如下:

\begin{enumerate}
    \item 为全局阈值T选择一个初始估计值,这里用了图像的平均灰度值作为初值。
    \item 利用T分割图像,产生两组像素:G1由灰度值大于T的所有像素组成,G1由所有小于等于T的像素组成。
    \item 计算G1、G2的平均灰度值m1、m2。
    \item 得到新的阈值$\displaystyle T_{new} = \frac{1}{2}(m1+m2)$
    \item 重复步骤2到步骤4,直到T与T\_new的差值小于一个预定的值。
\end{enumerate}


\section{代码实现}
本次实验编程语言选择的是Matlab。

\lstinputlisting[
    language  =   matlab
]{第五次/lab_new.m}

\section{实验结果}
实验结果如下:

\begin{figure}[H]
    \centering
    \includegraphics[width = 0.4\textwidth]{第五次/lab5-1.jpg}
    \caption{原始图像}
\end{figure}

\begin{figure}[H]
    \centering
    \includegraphics[width = 0.5\textwidth]{第五次/lab5-2.jpg}
    \caption{原始图像直方图}
\end{figure}

\begin{figure}[H]
    \centering
    \includegraphics[width = 0.4\textwidth]{第五次/lab5-3.jpg}
    \caption{全局阈值处理后的图像}
\end{figure}


\section{总结}
本次实验主要是通过Matlab编程语言实现了课程中所讲过的全局阈值处理。

实验还是比较简单,但是基本的全局阈值处理对原始图像的要求高,需要原始图像的直方图有明显的波谷。

\end{document}



