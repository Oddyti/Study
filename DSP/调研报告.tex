\documentclass{../source/Paper}

\major{信息工程}
\name{}
\articletitle{从傅立叶的生平思考我国的科教发展}
\stuid{}
\college{信息与电子工程学院}
\date{\today}
\course{数字信号处理}
\instructor{徐元欣}
%摘要
\Abstract{傅立叶级数以及傅立叶变换是信号与系统课程的核心,也是数字信号处理课程的基石,更是处理科学和工程诸多问题不可或缺的理论工具。除了傅立叶级数理论对于数学和科学的发展贡献外,傅立叶这位伟大的数学家的生平经历也具有很大的调研价值。本文希望通过调研法国数学家傅立叶的生平经历,以及提出傅立叶级数的过程,分析傅立叶有如此成功的原因,借此反思我国近代落后的原因以及思考我国实施科教兴国战略的意义与价值。}
%关键词
\Keyword{傅立叶、数字信号处理、科教兴国、法国大革命}

\begin{document}
\makeheader

广义来说,数字信号处理是研究用数字方法对信号进行分析、变换、滤波、检测、调制、解调以及快速算法的一门技术学科。另一个方面来说,数字信号处理主要是研究有关数字铝箔级数、离散变换快速算法和谱分析方法。而随着数字电路以及计算机技术的发展,数字新少处理级数也相应地得到发展,应用在了非常广泛的领域。

傅立叶级数以及傅立叶变换是信号与系统课程的核心,通过研究傅立叶的生平,可以帮助我们更好的理解他成功的原因,分析我国科教兴国战略的意义。

\section{傅立叶的生平}

傅立叶出生于欧塞尔(现为法国约讷省),是一位裁缝的儿子。他在九岁时成为孤儿。傅立叶被推荐给欧塞尔主教,通过这次介绍,他接受了圣马可修道院本笃会的教育。他本想加入军队科学兵团,但是由于傅立叶出身平民,而这个兵团的委任是为那些出身良好的贵族保留的,因此他没有资格加入。之后,他接受了数学的军事讲座。傅立叶在自己的选区中为推动法国大革命发挥了重要作用,在当地的革命委员会任职。1795年,傅立叶在师范学院任职,随后在巴黎综合理工学院接替约瑟夫-路易·拉格朗日的工作。

傅立叶于1798年陪同拿破仑·波拿巴进行埃及探险,担任科学顾问,并被任命为埃及研究所的秘书。在英国舰队阻隔了他们返回法国的道路后,傅立叶组织建立了了法国军队所需要的工厂车间,为法国军队提供战争弹药。他还向拿破仑在开罗创立的埃及研究所(也称为开罗研究所)撰写了几篇数学论文,以削弱英国在东方的影响力。1801年,英军取得胜利,法国在梅努将军的带领下下投降后,傅立叶回到了法国。

1801年,拿破仑被任命为格勒诺布尔伊泽尔省长(总督),负责监督道路建设和其他项目。然而,傅立叶在从埃及回来后,希望回巴黎并希望恢复他在巴黎综合理工学院教授的学术职位。但当时拿破仑在他的备注中写道:

“...伊泽尔省省长最近去世了,我想通过任命他到这个地方来表达我对公民傅立叶的信任。”

因此,出于对拿破仑的忠诚,傅立叶接任了省长一职。

在格勒诺布尔期间,他开始试验热量的传播。1807年12月21日,他向巴黎研究所提交了他的论文《论热量在固体中的传播》。1822年,傅立叶接替让·巴蒂斯特·约瑟夫·德朗布尔成为法国科学院常务秘书。1830年,他被选为瑞典皇家科学院的外籍院士。1830年,他的健康状况开始恶化,此后不久,他于1830年5月16日在床上去世。

傅立叶被埋葬在巴黎的拉雪兹神父公墓,这是一座装饰有埃及图案的坟墓,以反映他作为开罗研究所秘书的地位,以及他对《埃及描述》的整理。他的名字被刻在了埃菲尔铁塔上。

1849年在欧塞尔竖立了一座傅立叶的青铜雕像,但在第二次世界大战期间这个雕像被熔化用于军备。格勒诺布尔的约瑟夫·傅立叶大学就是以他的名字命名的。


\section{傅立叶变换的提出}

正是在格勒诺布尔,傅立叶进行了关于热量扩散的实验,随后,傅立叶在《热的分析理论》上发表了他关于热流的著作,其中他的推理基于牛顿的冷却定律,即两个相邻分子之间的热流与它们温度的极小差异成正比。这本书在56年后由弗里曼(1878年)翻译成英文,并附有社论上的“更正”。

这项工作有三个重要的贡献,一个纯粹是数学的,两个基本上是物理的。在数学中,傅里叶声称变量的任何函数,无论是连续的还是不连续的,都可以在变量的倍数的一系列正弦中展开。虽然这个结果在没有附加条件的情况下是不正确的,但傅立叶观察到一些不连续函数是无穷级数之和是一个突破。确定傅里叶级数何时收敛的问题几个世纪以来一直是基础问题。约瑟夫-路易·拉格朗日(Joseph-Louis
Lagrange)给出了这个(假)定理的特殊情况,并暗示这种方法是通用的,但他没有追求这个主题。彼得·古斯塔夫·勒琼·狄利克雷是第一个在一些限制性条件下令人满意地证明这一点的人。这项工作为今天所谓的傅里叶变换奠定了基础。

书中一个重要的物理贡献是方程中维度均匀性的概念;即,只有当等式的维度在等式的任何一侧匹配时,方程才能正式正确;傅立叶对维数分析做出了重要贡献。
另一个物理贡献是傅立叶提出的关于热传导扩散的偏微分方程。这个方程现在被教给每个数学物理的学生。


\section{结合傅立叶发展思考中国近代落后的原因}

傅立叶出生于1768年,他提出傅立叶级数的时候为1807年,这段时间正是法国大革命的时期,而傅立叶可以有如此的成就,与当时他处于的环境有很大的关系。

1789年,法国资产阶级以启蒙思想和理性主义为旗帜,⾼举``⾃由、平等、博爱''的⼤旗,领导了⼀场轰轰烈烈的⾰命运动。法国⼤⾰命后,⾸次确⽴了较为⾃由的制度和较为完善的政体,尤其是三权分⽴原则的确⽴以及标志着⼈的解放的《⼈权宣⾔》的发表,使⾃由、平等、民主的观念深⼊⼈⼼,同时宣告了资产阶级最基本的⼈权与公民权原则,为⾃然科学的⾃由探索提供了保障。众所周知,科学是创造性的活动,只有民主社会,才能提供这种⾃由:科学本性是实事求是,是按客观规律办事,除了科学家⾃⾝的努⼒之外,还需要民主制度作保障。

法国⼤⾰命对科学的重⼤贡献之⼀就是创造了崭新的资产阶级科学教育制这是⾃⽂艺复兴以来历次教育改⾰的成果,也是现代资产阶级科学教育制度的楷模。法国⼤⾰命破除了封建教育制度的陈腐法规,⼤兴资产阶级科学教育之风,吸收⾃⽂艺复兴以来历次警!鳘⾰的成果,创建了崭新的资产阶级科学教亨体制,不仅揭开了教育近代化的序幕,开创了近代⾼等教育的模式之⼀,⽽且旨次引⼊近代科学教育的内容,并创⽴科研机构辅助科学教育的发展,由此确⽴了完善的近代科教体制,并为科警发展培养了⼤批科学⼈才,形成了⼀⼤批⽣机勃勃的科家集团。这也是法国天⾰命对科技发展的⼜⼀⼤贡献。

法国新兴资产阶级在进⾏教育改⾰的同时,也按照⾃⼰的需要,对王家科学院进⾏了资产阶级的改造。1806年,拿破颁布建⽴帝国⼤学的法令,教育更加普及。专利制度、奖⾦制度相继确⽴,学术交流会、技术博览会层出不穷,基本上奠定了现代资本主义科教制度的基础。

不仅如此,法国⼤⾰命中还确⽴了“带薪式”的资产阶级科研制度、“科学院制度”,它们既可称为教育史上的创新,⽴都有⼒的⿎励科学家从事教育事业。

从这些可以看到,一个国家的科技想要发展,首先离不开的是人民的思想启蒙,法国大革命之前,已经有启蒙运动这样的运动,使得人民的思想得到了解放。其次是离不开对于教育的重视,政策上的教育改革与科技运行制度的改革,使得法国许多人才得以获得知识,得到教育。使得科学人才有用武之地,有专研发展的动力和环境。而这些恰恰是近代中国所或缺的,也是近代中国落后的原因之一。


\section{中国实施科教兴国战略的启示和思考}

通过分析傅立叶的成果原因,我们可以知道,一个国家要想发展与强大,良好以及普及教育和对于科学技术的重视是不可或缺的,而我国的科教兴国的战略提出正朝着这个方向做出了很大的迈进。

“科教兴国”思想的理论基础是邓小平同志关于科学技术是第一生产力的思想。1978年,邓小平在科学和教育工作座谈会上提出:“我们国家要赶上世界先进水平,从何着手呢?我想,要从科学和教育着手”,“不抓科学、教育,四个现代化就没有希望,就成为一句空话”,明确把科教发展作为发展经济、建设现代化强国的先导,摆在我国发展战略的首位。从70年代后期到90年代初期,邓小平同志坚持“实现四个现代化,科学技术是关键,基础是教育”的核心思想,为“科教兴国”发展战略的形,成奠定了坚实的理论和实践基础。

科教兴国战略实施有着重大的意义。首先,科学技术是第一生产力,科技发展是经济发展的决定性因素,社会主义的根本任务是解放和发展生产力。实施科教兴国战略,有助于实现两个根本性转变,调整、优化产业结构,培育新的经济增长点,提高企业经济效益,开拓新的市场空间,极大地促进生产力的发展。其次,我国政府实施科教兴国战略,是行使组织和领导社会主义经济建设、精神文明建设等职能的表现。国家实施这些职能,有利于促进经济的发展和社会的全面进步,提高人民的物质文化生活水平,实现社会主义的目的、目标;有利于巩固和发展社会主义制度,实现国家和社会的稳定;有利于增强我国的综合国力,提高我国的国际地位。

\end{document}