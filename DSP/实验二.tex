\documentclass{../source/Experiment copy}

\major{信息工程}
\name{}
\title{DFT/FFT的应用之一——确定性信号谱分析}
\stuid{}
\college{信息与电子工程学院}
\date{\today}
\lab{——}
\course{数字信号处理}
\instructor{徐元欣}
\grades{}
\expname{DFT/FFT的应用之一——确定性信号谱分析}
\exptype{验证}
\partner{——}
\begin{document}
\makeheader
\section{实验目的和要求}
谱分析即求信号的频谱。本实验采用DFT/FFT技术对周期性信号进行谱分析。通过实验,了解用X(k)近似地表示频谱$\rm  X(ej\omega)$带来的栅栏效应、混叠现象和频谱泄漏,了解如何正确地选择参数(抽样间隔T、抽样点数N)。

\section{实验内容和步骤}
2-1 \,  \,  \, 选用最简单的周期信号:单频正弦信号、频率f=50赫兹,进行谱分析。

2-2 \,  \,  \, 谱分析参数可以从下表中任选一组(也可自定)。对各组参数时的序列,计算:一个正弦周期是否对应整数个抽样间隔?观察区间是否对应整数个正弦周期?
\begin{table}[H]
    \centering
    \begin{tabular}{|c|c|l|c|}
        \hline
        {信号频率f(赫兹)} & {谱分析参数} & \multicolumn{1}{c|}{抽样间隔T} & {截断长度N}  \\
        {}                  & {}           & \multicolumn{1}{c|}{(秒)}    & (抽样个数) \\ \hline
        50                  & 第一组参数   & 0.000625                       & 32           \\ \hline
        50                  & 第二组参数   & 0.005                          & 32           \\ \hline
        50                  & 第三组参数   & 0.0046875                      & 32           \\ \hline
        50                  & 第四组参数   & 0.004                          & 32           \\ \hline
        50                  & 第五组参数   & 0.0025                         & 16           \\ \hline
    \end{tabular}
\end{table}
2-3 \,  \,  \, 对以上几个正弦序列,依次进行以下过程。

2-3-1 \, 观察并记录一个正弦序列的图形(时域)、频谱(幅度谱、频谱实 部、频谱虚部)形状、幅度谱的第一个峰的坐标(U,V)。

2-3-2 \, 分析抽样间隔T、截断长度N(抽样个数)对谱分析结果的影响;
2-3-3 \, 思考X(k)与$\rm  X(ej\omega)$的关系;

2-3-4 \, 讨论用X(k)近似表示$\rm  X(ej\omega)$时的栅栏效应、混叠现象、频谱泄漏。


\section{主要仪器设备}

MATLAB编程。

\section{操作方法和实验步骤}

(参见“二、实验内容和步骤”)

\section{实验数据记录和处理}

MATLAB程序清单
\subsection{主函数}
\lstinputlisting[
    language       =   Matlab,
    title     =   {主函数}
]{src/exp2.m}
\subsection{绘图函数}
\lstinputlisting[
    language       =   Matlab,
    title     =   {绘图函数}
]{src/myPlot2.m}
\subsection{DFT函数}
\lstinputlisting[
    language       =   Matlab,
    title     =   {DFT函数}
]{src/myDFT.m}
\section{实验结果与分析}
\subsection{情况预测}
实验前预习有关概念,并根据上列参数来推测相应频谱的形状、谱峰所在频率(U)和谱峰的数值(V)、混叠现象和频谱泄漏的有无。
\begin{table}[H]
    \centering
    \begin{tabular}{|c|c|l|c|c|c|}
        \hline
        信号频率f & 谱分析参数 & \multicolumn{1}{c|}{抽样间隔T} & 截断长度N    & 谱峰所在频率 & 谱峰的数值 \\
        (赫兹)  &            & \multicolumn{1}{c|}{(秒)}    & (抽样个数) & (U)        & (V)      \\ \hline
        50        & 第一组参数 & 0.000625                       & 32           & 1            & 16         \\ \hline
        50        & 第二组参数 & 0.005                          & 32           & 8            & 16         \\ \hline
        50        & 第三组参数 & 0.0046875                      & 32           & 7            & 10.25      \\ \hline
        50        & 第四组参数 & 0.004                          & 32           & 6            & 12         \\ \hline
        50        & 第五组参数 & 0.0025                         & 16           & 2            & 8          \\ \hline
    \end{tabular}
\end{table}

满足内奎斯特定律时不会产出混叠现象,即采样频类需要大于或等于信号最高频率的两倍。实验中也即采样周期小于等于0.01s则可满足奈奎斯特定律。所以五组实验中,都满足奈奎斯特定律。
\begin{table}[H]
    \centering
    \begin{tabular}{|c|c|l|c|c|}
        \hline
        信号频率f & 谱分析参数 & \multicolumn{1}{c|}{抽样间隔T} & \{截断长度N\} & 区间包括正弦周期个数 \\
        (赫兹)  &            & \multicolumn{1}{c|}{(秒)}    & (抽样个数)  &                      \\ \hline
        50        & 第一组参数 & 0.000625                       & 32            & 1                    \\ \hline
        50        & 第二组参数 & 0.005                          & 32            & 8                    \\ \hline
        50        & 第三组参数 & 0.004688                       & 32            & 7.5                  \\ \hline
        50        & 第四组参数 & 0.004                          & 32            & 6.4                  \\ \hline
        50        & 第五组参数 & 0.0025                         & 16            & 2                    \\ \hline
    \end{tabular}
\end{table}
如上表,当采样长度也就是窗函数长度为采样周期的整数倍,则不会出现频谱泄露,所以推测出第三、四组会出现频谱泄露。

\subsection{实验结果记录}
观察实验结果(数据及图形)的特征,做必要的记录。
\subsubsection{第一组参数}
\begin{figure}[H]
    \centering
    \includegraphics[width = \textwidth]{src/exp2_1_1.png}
\end{figure}

\begin{figure}[H]
    \centering
    \includegraphics[width = \textwidth]{src/exp2_1_2.png}
\end{figure}

\begin{figure}[H]
    \centering
    \includegraphics[width = \textwidth]{src/exp2_1_3.png}
    \caption{第一组:f = 50Hz, T = 0.000625s, N = 32}
\end{figure}

\subsubsection{第二组参数}
\begin{figure}[H]
    \centering
    \includegraphics[width = \textwidth]{src/exp2_2_1.png}
\end{figure}

\begin{figure}[H]
    \centering
    \includegraphics[width = \textwidth]{src/exp2_2_2.png}
\end{figure}

\begin{figure}[H]
    \centering
    \includegraphics[width = \textwidth]{src/exp2_2_3.png}
    \caption{第二组:f = 50Hz, T = 0.005s, N = 32}
\end{figure}

\subsubsection{第三组参数}
\begin{figure}[H]
    \centering
    \includegraphics[width = \textwidth]{src/exp2_3_1.png}
\end{figure}

\begin{figure}[H]
    \centering
    \includegraphics[width = \textwidth]{src/exp2_3_2.png}
\end{figure}

\begin{figure}[H]
    \centering
    \includegraphics[width = \textwidth]{src/exp2_3_3.png}
    \caption{第三组:f = 50Hz, T = 0.0046875s, N = 32}
\end{figure}
可以看见,此组出现了频谱泄露的现象。
\subsubsection{第四组参数}
\begin{figure}[H]
    \centering
    \includegraphics[width = \textwidth]{src/exp2_4_1.png}
\end{figure}

\begin{figure}[H]
    \centering
    \includegraphics[width = \textwidth]{src/exp2_4_2.png}
\end{figure}

\begin{figure}[H]
    \centering
    \includegraphics[width = \textwidth]{src/exp2_4_3.png}
    \caption{第四组:f = 50Hz, T = 0.004s, N = 32}
\end{figure}
可以看见,此组出现了频谱泄露的现象。
\subsubsection{第五组参数}
\begin{figure}[H]
    \centering
    \includegraphics[width = \textwidth]{src/exp2_5_1.png}
\end{figure}

\begin{figure}[H]
    \centering
    \includegraphics[width = \textwidth]{src/exp2_5_2.png}
\end{figure}

\begin{figure}[H]
    \centering
    \includegraphics[width = \textwidth]{src/exp2_5_3.png}
    \caption{第五组:f = 50Hz, T = 0.0025s, N = 16}
\end{figure}
结果分析
\begin{table}[H]
    \centering
    \begin{tabular}{|c|c|c|c|c|c|c|}
        \hline
        组别       & 抽样间隔T & 截断长度N    & 谱峰所在频率 & 谱峰的数值 & 是否混叠 & 是否频谱泄露 \\
                   & (秒)    & (抽样个数) & (U)        & (V)      &          &              \\ \hline
        第一组参数 & 0.000625  & 32           & 1            & 16         & 否       & 否           \\ \hline
        第二组参数 & 0.005     & 32           & 8            & 16         & 否       & 否           \\ \hline
        第三组参数 & 0.0046875 & 32           & 7            & 10.2519    & 否       & 是           \\ \hline
        第四组参数 & 0.004     & 32           & 6            & 11.965     & 否       & 是           \\ \hline
        第五组参数 & 0.0025    & 16           & 2            & 8          & 否       & 否           \\ \hline
    \end{tabular}
\end{table}
\subsection{现象解释}
用基本理论、基本概念来解释各种现象。

\subsubsection{频域混叠}
序列的频谱是被采样信号频谱的周期延拓,当采样频率不满足奈奎斯特定律时,也就是不能满足采样频率大于等于两倍的信号最高频率时,就会发生频谱混叠,使得采样后的信号序列频谱不能真实的反映原信号的频谱。

\subsubsection{频谱泄露}

采样后,对序列进行了截断,等同于于乘上了一个窗函数,在频谱上,相当于频域上与sinc函数进行卷积,因此采样后的信号总是存在高频分量,因此总是存在频域混叠的现象,也会存在频域泄露的现象。

而如果DFT采集时间窗口内的信号的周期延拓与实际信号完全吻合,那么就不会出现泄漏现象。换句话说,对于周期信号,如果采集时间窗口内正好包含整数个信号周期,就能避免频谱泄漏。

所以第一、二、五组不会出现频谱泄露的现象,第三、四组会出现频谱泄漏。

\end{document}