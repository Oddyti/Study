\documentclass{../source/Experiment}

\major{信息工程}
\name{姚桂涛}
\title{DFT/FFT的应用之一——确定性信号谱分析}
\stuid{3190105597}
\college{信息与电子工程学院}
\date{\today}
\lab{——}
\course{数字信号处理}
\instructor{徐元欣}
\grades{}
\expname{DFT/FFT的应用之一——确定性信号谱分析}
\exptype{验证}
\partner{——}
\begin{document}
    \makeheader
    \section{实验目的和要求}
    谱分析即求信号的频谱。本实验采用DFT/FFT技术对周期性信号进行谱分析。通过实验,了解用X(k)近似地表示频谱$\rm  X(ej\omega)$带来的栅栏效应、混叠现象和频谱泄漏,了解如何正确地选择参数(抽样间隔T、抽样点数N)。

    \section{实验内容和步骤}
    2-1 \,  \,  \, 选用最简单的周期信号:单频正弦信号、频率f=50赫兹,进行谱分析。

    2-2 \,  \,  \, 谱分析参数可以从下表中任选一组(也可自定)。对各组参数时的序列,计算:一个正弦周期是否对应整数个抽样间隔?观察区间是否对应整数个正弦周期?
    \begin{table}[H]
        \centering
        \begin{tabular}{|c|c|l|c|}
        \hline
        \textbf{信号频率f(赫兹)} & \textbf{谱分析参数} & \multicolumn{1}{c|}{\textbf{抽样间隔T}} & \textbf{截断长度N} \\
        \textbf{}          & \textbf{}      & \multicolumn{1}{c|}{(秒)}            & (抽样个数)         \\ \hline
        50                 & 第一组参数          & 0.000625                            & 32             \\ \hline
        50                 & 第二组参数          & 0.005                               & 32             \\ \hline
        50                 & 第三组参数          & 0.0046875                           & 32             \\ \hline
        50                 & 第四组参数          & 0.004                               & 32             \\ \hline
        50                 & 第五组参数          & 0.0025                              & 16             \\ \hline
        \end{tabular}
    \end{table}
    2-3 \,  \,  \, 对以上几个正弦序列,依次进行以下过程。

    2-3-1 \, 观察并记录一个正弦序列的图形(时域)、频谱(幅度谱、频谱实 部、频谱虚部)形状、幅度谱的第一个峰的坐标(U,V)。

    2-3-2 \, 分析抽样间隔T、截断长度N(抽样个数)对谱分析结果的影响;
    2-3-3 \, 思考X(k)与$\rm  X(ej\omega)$的关系;

    2-3-4 \, 讨论用X(k)近似表示$\rm  X(ej\omega)$时的栅栏效应、混叠现象、频谱泄漏。


    \section{主要仪器设备}
    
    MATLAB编程。

    \section{操作方法和实验步骤}

    (参见“二、实验内容和步骤”)

    \section{实验数据记录和处理}
    
        MATLAB程序清单
        \subsection{各序列主函数}
            
    \section{实验结果与分析}
        \subsection{序列各图像特征与解释}

            
 
\end{document}