\documentclass{../source/Experiment}

\major{信息工程}
\name{姚桂涛}
\title{有限长序列、频谱、DFT的性质}
\stuid{3190105597}
\college{信息与电子工程学院}
\date{\today}
\lab{——}
\course{数字信号处理}
\instructor{徐元欣}
\grades{}
\expname{有限长序列、频谱、DFT的性质}
\exptype{演示}
\partner{——}
\begin{document}
    \makeheader
    \section{实验目的和要求}

    FFT是快速计算DFT的一类算法的总称。通过序列分解,用短序列的DFT代替长序列的DFT,使得计算量大大下降。基4-FFT是混合基FFT的一个特例。

    通过编写基4-FFT算法程序,加深对FFT思路、算法结构的理解。


    \section{实验内容和步骤}
    编写16点基4-FFT算法的MATLAB程序(studentname.m文件)。

    产生16点输入序列x,用自己的学号作为前10点的抽样值,后面补6个零值抽样。算出16点频谱序列X,用stem(X)显示频谱图形。

    \section{主要仪器设备}

    MATLAB编程。

    \section{操作方法和实验步骤}

    (参见“二、实验内容和步骤”)

\end{document}