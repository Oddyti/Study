\documentclass{../source/Experiment}

\major{信息工程}
\name{姚桂涛}
\title{基于RV32I指令集的RISC-V微处理器设计}
\stuid{3190105597}
\college{信息与电子工程学院}
\date{\today}
\lab{教11-400}
\course{计算机组成与设计}
\instructor{屈民军、唐奕}
\grades{}
\expname{RISC-V微处理器设计}
\exptype{设计}
\partner{}
\begin{document}
    \makecover
    \makeheader
    \section{实验目的}
    
    \section{实验任务}
        \subsection{基本要求}
        设计一个流水线RISC-V微处理器,具体要求如下所述。

        (1) \, 至少运行下列RV32I核心指令。

        算术运算指令:add、sub、addi

        逻辑运算指令:and、or、xor、slt、sltu、di、ori、xori、slti、sltiu

        移位指令:sll、srl、sra、slli、srli、srai

        条件分支指令:beq、bne、blt、bge、bltu、eu

        无条件跳转指令:jal、jalr

        数据传送指令:lw、sw、lui、auipc
    
        空指令:nop

        (2) \, 采用 5 级流水线技术,对数据冒险实现转发或阻塞功能。

        (3) \, 在 Nexys Video 开发系统中实现RISC-V 微处理器, 要求 CPU 的运行速度大于 25MHz。
        \subsection{扩展要求}
        (1) \, 要求设计的微处理器还能运行lb、lh、ld、lbu、lhu、lwu、sb、sh 或 sd 等字节、半字和双字数据传送指令。

        (2) \, 要求设计的CPU 增加异常(exception)、自陷(trap)、中断(interrupt)等处理方案。
        
    \section{实验原理与模块设计}
        \subsection{总体设计}
        流水线是数字系统中一种提高系统稳定性和工作速度的方法,广泛应用于高档CPU 的架构中。根据RISC-V 处理器指令的特点,将指令整体的处理过程分为取指令(IF)、指令译码(ID)、执行(EX)、存储器访问(MEM)和寄存器回写(WB)五级。如图30. 2 示,一个指令的执行需要 5 个时钟周期,每个时钟周期的上升沿来临时,此指令所代表的一系列数据和控制信息将转移到下一级处理。

        图  所示为符合设计要求的流水线 RISC-V 微处理器的原理框图,采用五级流水线。由于在流水线中 ,数据和控制信息将在时钟周期的上升沿转移到下一级,所以规定流水线转移的变量命名遵守如下 格式:名称\_流水线级名称。如,在 ID 级指令译码电路(decode)产生的寄存器写允许信号 RegWrite 在ID 级、EX 级、MEM 级和WB 级上的命名分别为RegWrite\_id、RegWrite\_ex 、RegWrite\_mem 和RegWrite\_wb。在顶层文件中,类似的变量名称有近百个,这样的命名方式起到了很好的识别作用。

        \subsection{流水线RISC-V微处理器的设计}
        根据流水线不同阶段,将系统划分为 IF、ID、EX 和 MEM 四大模块,WB 部分功能电路非常简单,可直接在顶层文件中设计。另外,系统还包含 IF/ID、ID/EX、EX/MEM、MEM/WB 四个流水线寄存器。
        \subsubsection{取指令级模块(IF)的设计}
        IF 模块由指令指针寄存器(PC)、指令存储器子模块(Instruction ROM)、指令指针选择器(MUX)
        和一个 32 位加法器组成,IF 模块接口信息如表  所示。

        核心代码如下:

        其中
        
        \subsubsection{指令译码模块(ID)的设计}
        指令译码模块的主要作用是从机器码中解析出指令,并根据解析结果输出各种控制信号。ID 模块主要
        由指令译码(Decode)、寄存器堆(Registers)、冒险检测、分支检测和加法器等组成。ID 模块的接口信息如表  所示。
        
        (1) \, 寄存器堆(Register)子模块的设计

        寄存器堆由 32 个 32 位寄存器组成,这些寄存器通过寄存器号进行读写存取。寄存器堆的原理框图如图30.8 所示。因为读取寄存器不会更改其内容,故只需提供寄存号即可读出该寄存器内容。读取端口采用数据选择器即可实现读取功能。

        寄存器堆设计还应解决三阶数据相关的数据转发问题。当满足三阶数据相关条件时,寄存器具有Read After Write 特性。

        RBW子模块核心代码如下:

        实现Read After Write寄存堆代码如下:

        (2) \, 指令译码(包含立即数产生电路)子模块的设计

        该子模块主要作用是根据指令确定各个控制信号的值,同时产生立即数 Imm 和偏移量offet。该模块是一个组合电路。

        根据操作数的来源和立即数的不同,将指令细分为不同类型,并对于不同类型产生不同的控制信号。同时产生立即数和偏移量。

        核心代码如下:

        (3) \, 分支检测电路的设计

        分支检测电路主要用于判断分支条件是否成立,在 Verilog HDL 可以用比较运算符号“>”、“==” 和“<”描述,但要注意符号数和无符号数的处理方法不同。在这里,我们用加法器来实现。

        用一个 32 位加法器完成rs1Data+(~rs2Data)+1(即 rs1Data-rs2Data),设结果为 sum [31:0] 。

        确定比较运算的结果。对于比较运算来说,如果最高位不同,即rs1Data [31]  rs2Data [31],可根据rs1Data [31]、rs2Data [31]决定比较结果,但是应注意符号数、无符号数的最高位rs1Data [31]、rs2Data [31] 代表意义不同。若两数最高位相同,则两数之差不会溢出,所以比较运算结果可由两个操作数之差的符号位 sum[31]决定。

        最终得到符号数与无符号数比较运算的结果:

        $ \rm isLT=rs1Data[31]\&\&(~rs2Data[31]) || (rs1Data[31]~^rs2Data[31]) \&\& sum[31]$

        $ \rm isLTU= (~rs1Data[31] )\&\& rs2Data[31] || (rs1Data[31]~^rs2Data[31]) \&\& sum[31]$

        最后用数据选择器完成下式即完成分支检测。

        $$Branch = \left\{
            \begin{aligned}
             & ~ (| sum[31 : 0]); & SB\_type & \, \& \& (funct3 == beq\_funct3 )\\
             & | sum[31 : 0];  & SB\_type & \, \& \& (funct3 = bne\_funct3 )\\
             & isLT;  & SB\_type & \, \& \&(funct3 = blt\_funct3 )\\
             & ~ isLT ;  & SB\_type & \, \& \& (funct3 = bge\_funct3 )\\
             & isLTU ;  & SB\_type & \, \& \& (funct3 = bltu\_funct 3)\\
             & ~ isLTU ;  & SB\_type & \, \& \& (funct3 = bgeu\_funct3)\\
             & 0  & others  & 
            \end{aligned}\right.   
        $$

        \subsubsection{执行模块(EX)的设}

        \subsubsection{数据存储器模块(DataRAM)的设计}

        \subsubsection{流水线寄存器的设计}

        \subsubsection{顶层文件的设计}

    \subsection{实验设备}
        \begin{enumerate}
            \item  装有Vivado 和 ModelSim SE 软件的计算机。
            \item  Nexys Video 开发板一套。
            \item  带有HDMI 接口的显示器一台。
        \end{enumerate}

    \section{实验结果}
        



\end{document}